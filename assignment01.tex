\documentclass{article}
%\usepackage[utf8]{inputenc}
\usepackage{kotex}
\usepackage{graphicx}


\title{Assignment 01}
\author{20145708 양지혜}
\date{September 2018}

\begin{document}

\maketitle

\section{Git}
Git은 컴퓨터에서 파일의 변경사항을 추적하고 여러 명의 사용자가 파일을 간편하게 사용 할 수 있는 분산 버전 시스템이다.
\\
버전 관리 시스템이란 소스를 하나 또는 묶음을 하나의 버전으로 간주하여 관리하는 시스템을 말한다.
\\
주로 소프트 웨어 개발에서 코드관리를 위해 사용된다.
\\
같은 모듈을 개발하면서 소스를 서로 공유하면서 개발 할 수 있으며, 권한 설정을 위해 각 개발자 별로 가능한 소스 목록을 제어 할 수 있다.
\\
개별적으로 버전, 파일의 변경사항 등을 지속적으로 추적하기 위해 사용되고 있다.
\\
분산 버전 관리 시스템이기 때문에 빠른 속도에 중점을 두고 있으며, 중앙 서버와 독립적으로 동작할 수 있는 완전한 비절차적 기능을 가지고 있다.

\section{Git Hub}

깃 허브는 분산 관리 툴 인 깃을 사용할 수 있는 웹호스팅 서비스 이다.
\\
깃은 텍스트령어 입력 방식인데 반해 깃 허브는 그래픽 유저 인터 페이스를 제공한다. 
\\
또한 여러 유명한 오픈소스 라이브러리등을 공유하고 공유받을 수 있다.
\\
여러 질의 응답, 이슈를 기록하기 위한 issues 페이지를 제공한다. 
\\
\section{Git의 용어}

Pull : Git 저장소 서버로부터 내 컴퓨터 로컬로 전체 버전 정보를 가져온다.
\\
Commit : 추가/수정/삭제된폴더/파일들을 1개의 버전을 간주하고 내 컴퓨터 로컬에 버전 정보를 기록한다.
\\
Push : 내 컴퓨터 로컬에 저장되어 있는 정보를 Git 저장소 서버에 올린다.
\\
Branch : 버전들을 묶어서 Branch 라고 한다. 기본을 Master라고 한다.
\\
Merge : 버전관리 시 여러 개발자가 각자 버전을 개발한 버전을 합치거나, 서로 다른 branch를 하나로 합치는 경우를 Merge 라고 한다.

\section{Try to git at my local}
 
 
\maketitle

\includegraphics[width=0.8\textwidth]{assignment01/two.png}
\\
\includegraphics[width=0.8\textwidth]{assignment01/one.png}
\\
\\
Git을 설치하고 assignment01 파일을 local 에 생성했다.
\\
Git을 초기화 하고 README.md 파일을 생성하였다.
\\
git status 명령어를 사용하여 assignment01의 상태를 확인 할 수 있다.
\\
Git에 add 명령어를 사용하여 assignment01 폴더에 README.md 를 추가하고 commit 하였다.
\\
\\
\includegraphics[width=1.3\textwidth]{assignment01/6.png}
\\
\\
git remote add 명령어로 새 리모트 저장소를 생성하였다.
\\
git remote 명령으로 현재 프로젝트에 등록된 리모트 저장소를 확인 할 수 있다.
\\
\\
\includegraphics[width=0.7\textwidth]{assignment01/commit.png}
\\
commit 명령어를 사용하였다. commit할 데이터가 없을 경우 이와 같이 나온다.
\\
\\
\includegraphics[width=0.8\textwidth]{assignment01/pull.png}
\\
\\
pull 명령어를 사용하여 Git 저장소에서 내 컴퓨터 로컬로 데이터를 가져왔다.
\\
\\
\includegraphics[width=0.8\textwidth]{assignment01/ppppppush.png}
\\
\\
push 명령어를 사용하여 내 컴퓨터의 저장되어 있는 정보를 Git 저장소에 저장하였다.
\\
\\
\includegraphics[width=0.8\textwidth]{assignment01/5.png}
\\
\\
다시 push 명령어를 사용하였을 경우 모두 업데이트가 완료 되었다는 메세지가 뜬다.


\end{document}


